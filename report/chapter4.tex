\chapter{SYSTEM DESIGN}

\section{System Architecture Diagram}

The Expense Manager system follows a three-tier architecture pattern that separates the presentation layer, business logic layer, and data access layer. This architecture ensures scalability, maintainability, and security.

\begin{figure}[H]
\centering
\includegraphics[width=0.6\textwidth]{images/architecture.png}
\caption{System Architecture Diagram}
\label{fig:architecture}
\end{figure}

\section{Use Case Diagram}

The use case diagram illustrates the functional requirements and interactions between users and the system.

\begin{figure}[H]
\centering
\includegraphics[width=0.3\textwidth]{images/usecase.png}
\caption{Use Case Diagram}
\label{fig:usecase}
\end{figure}

\section{Class Diagram}

The class diagram shows the static structure of the system including classes, attributes, methods, and relationships.

\begin{figure}[H]
\centering
\includegraphics[width=1\textwidth]{images/classdiagram.png}
\caption{Class Diagram}
\label{fig:class}
\end{figure}

\section{Activity Diagram}

The activity diagram shows the workflow of adding a new expense transaction.

\begin{figure}[H]
\centering
\includegraphics[width=0.8\textwidth]{images/activity.png}
\caption{Activity Diagram - Add Expense}
\label{fig:activity}
\end{figure}

\section{Data Flow Diagram}

\subsection{Level 0 DFD}

The Level 0 DFD provides a high-level overview of the Expense Manager system, showing the main data flows between the user and the system.

\begin{figure}[H]
\centering
\includegraphics[width=0.8\textwidth]{images/dfd0.png}
\caption{Level 0 Data Flow Diagram}
\label{fig:dfd0}
\end{figure}

\subsection{Level 1 DFD}

\begin{figure}[H]
\centering
\includegraphics[width=1\textwidth]{images/dataflow.png}
\caption{Level 1 Data Flow Diagram}
\label{fig:dfd1}
\end{figure}
