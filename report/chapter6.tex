\chapter{TESTING}

\section{Testing Overview}

Testing is a critical phase in the software development lifecycle that ensures the system meets specified requirements and functions correctly under various conditions. The Expense Manager system underwent comprehensive testing at multiple levels to validate functionality, performance, and security.

\section{Testing Methodology}

The testing approach follows a systematic methodology covering:
\begin{itemize}
    \item Unit Testing: Testing individual components in isolation
    \item Integration Testing: Testing component interactions
    \item System Testing: Testing the complete integrated system
    \item User Acceptance Testing: Validating system meets user requirements
\end{itemize}

\section{Unit Testing}

Unit testing focuses on testing individual methods and classes to ensure they function correctly in isolation.

\subsection{Test Case 1: User Authentication}
\begin{table}[h]
\centering
\begin{tabular}{|p{3cm}|p{10cm}|}
\hline
\textbf{Test Case ID} & UT001 \\
\hline
\textbf{Test Case Name} & User Login Validation \\
\hline
\textbf{Objective} & Verify user authentication with valid credentials \\
\hline
\textbf{Preconditions} & User account exists in database \\
\hline
\textbf{Test Steps} & 1. Call loginUser() method with valid username and password\\
& 2. Verify password hashing\\
& 3. Check database query execution \\
\hline
\textbf{Expected Result} & Method returns User object with correct details \\
\hline
\textbf{Actual Result} & User object returned successfully \\
\hline
\textbf{Status} & PASS \\
\hline
\end{tabular}
\caption{Unit Test Case - User Authentication}
\end{table}

\subsection{Test Case 2: Password Hashing}
\begin{table}[h]
\centering
\begin{tabular}{|p{3cm}|p{10cm}|}
\hline
\textbf{Test Case ID} & UT002 \\
\hline
\textbf{Test Case Name} & Password Hash Generation \\
\hline
\textbf{Objective} & Verify SHA-256 password hashing functionality \\
\hline
\textbf{Preconditions} & None \\
\hline
\textbf{Test Steps} & 1. Call hashPassword() with test password\\
& 2. Verify hash length and format\\
& 3. Test with same password multiple times \\
\hline
\textbf{Expected Result} & Consistent 64-character SHA-256 hash generated \\
\hline
\textbf{Actual Result} & Hash generated correctly and consistently \\
\hline
\textbf{Status} & PASS \\
\hline
\end{tabular}
\caption{Unit Test Case - Password Hashing}
\end{table}

\subsection{Test Case 3: Expense Addition}
\begin{table}[h]
\centering
\begin{tabular}{|p{3cm}|p{10cm}|}
\hline
\textbf{Test Case ID} & UT003 \\
\hline
\textbf{Test Case Name} & Add Expense Transaction \\
\hline
\textbf{Objective} & Verify expense addition to database \\
\hline
\textbf{Preconditions} & Valid user ID and database connection \\
\hline
\textbf{Test Steps} & 1. Create Expense object with valid data\\
& 2. Call addExpense() method\\
& 3. Verify database insertion \\
\hline
\textbf{Expected Result} & Method returns true and expense saved to database \\
\hline
\textbf{Actual Result} & Expense successfully added to database \\
\hline
\textbf{Status} & PASS \\
\hline
\end{tabular}
\caption{Unit Test Case - Expense Addition}
\end{table}

\section{Integration Testing}

Integration testing verifies the interaction between different system components.

\subsection{Test Case 4: User Registration Flow}
\begin{table}[h]
\centering
\begin{tabular}{|p{3cm}|p{10cm}|}
\hline
\textbf{Test Case ID} & IT001 \\
\hline
\textbf{Test Case Name} & Complete User Registration Process \\
\hline
\textbf{Objective} & Test integration between registration servlet and UserDAO \\
\hline
\textbf{Preconditions} & Database connection available \\
\hline
\textbf{Test Steps} & 1. Submit registration form with valid data\\
& 2. Verify servlet processes request\\
& 3. Check UserDAO creates user record\\
& 4. Verify password hashing integration \\
\hline
\textbf{Expected Result} & User successfully registered and redirected to login \\
\hline
\textbf{Actual Result} & Registration completed successfully \\
\hline
\textbf{Status} & PASS \\
\hline
\end{tabular}
\caption{Integration Test Case - User Registration}
\end{table}

\subsection{Test Case 5: Transaction Management Integration}
\begin{table}[h]
\centering
\begin{tabular}{|p{3cm}|p{10cm}|}
\hline
\textbf{Test Case ID} & IT002 \\
\hline
\textbf{Test Case Name} & Expense Addition with Session Validation \\
\hline
\textbf{Objective} & Test integration between servlet, session, and DAO \\
\hline
\textbf{Preconditions} & User logged in with valid session \\
\hline
\textbf{Test Steps} & 1. Submit expense form\\
& 2. Verify session validation\\
& 3. Check servlet processes request\\
& 4. Verify DAO saves expense \\
\hline
\textbf{Expected Result} & Expense saved and user redirected to dashboard \\
\hline
\textbf{Actual Result} & Integration working correctly \\
\hline
\textbf{Status} & PASS \\
\hline
\end{tabular}
\caption{Integration Test Case - Transaction Management}
\end{table}

\section{System Testing}

System testing validates the complete integrated system against specified requirements.

\subsection{Test Case 6: End-to-End User Workflow}
\begin{table}[h]
\centering
\begin{tabular}{|p{3cm}|p{10cm}|}
\hline
\textbf{Test Case ID} & ST001 \\
\hline
\textbf{Test Case Name} & Complete User Journey \\
\hline
\textbf{Objective} & Test complete user workflow from registration to reporting \\
\hline
\textbf{Preconditions} & System deployed and database accessible \\
\hline
\textbf{Test Steps} & 1. Register new user account\\
& 2. Login with credentials\\
& 3. Add multiple expenses and income\\
& 4. View dashboard summary\\
& 5. Generate and export reports \\
\hline
\textbf{Expected Result} & All operations complete successfully with correct data \\
\hline
\textbf{Actual Result} & Complete workflow executed successfully \\
\hline
\textbf{Status} & PASS \\
\hline
\end{tabular}
\caption{System Test Case - End-to-End Workflow}
\end{table}

\subsection{Test Case 7: Concurrent User Access}
\begin{table}[h]
\centering
\begin{tabular}{|p{3cm}|p{10cm}|}
\hline
\textbf{Test Case ID} & ST002 \\
\hline
\textbf{Test Case Name} & Multiple User Concurrent Access \\
\hline
\textbf{Objective} & Verify system handles multiple simultaneous users \\
\hline
\textbf{Preconditions} & Multiple user accounts available \\
\hline
\textbf{Test Steps} & 1. Login 10 users simultaneously\\
& 2. Perform transactions concurrently\\
& 3. Verify data isolation\\
& 4. Check system performance \\
\hline
\textbf{Expected Result} & System maintains performance and data integrity \\
\hline
\textbf{Actual Result} & System handled concurrent access successfully \\
\hline
\textbf{Status} & PASS \\
\hline
\end{tabular}
\caption{System Test Case - Concurrent Access}
\end{table}

\subsection{Test Case 8: Security Testing}
\begin{table}[h]
\centering
\begin{tabular}{|p{3cm}|p{10cm}|}
\hline
\textbf{Test Case ID} & ST003 \\
\hline
\textbf{Test Case Name} & SQL Injection Prevention \\
\hline
\textbf{Objective} & Verify system prevents SQL injection attacks \\
\hline
\textbf{Preconditions} & System running with database connection \\
\hline
\textbf{Test Steps} & 1. Attempt SQL injection in login form\\
& 2. Try malicious input in expense description\\
& 3. Test special characters in all input fields\\
& 4. Verify prepared statements protection \\
\hline
\textbf{Expected Result} & All injection attempts blocked, no data corruption \\
\hline
\textbf{Actual Result} & System successfully prevented all injection attempts \\
\hline
\textbf{Status} & PASS \\
\hline
\end{tabular}
\caption{System Test Case - Security Testing}
\end{table}

\section{Performance Testing}

Performance testing evaluates system behavior under various load conditions.

\subsection{Test Case 9: Response Time Testing}
\begin{table}[h]
\centering
\begin{tabular}{|p{3cm}|p{10cm}|}
\hline
\textbf{Test Case ID} & PT001 \\
\hline
\textbf{Test Case Name} & Page Load Response Time \\
\hline
\textbf{Objective} & Measure system response times under normal load \\
\hline
\textbf{Preconditions} & System deployed with test data \\
\hline
\textbf{Test Steps} & 1. Measure login page load time\\
& 2. Test dashboard rendering time\\
& 3. Check report generation time\\
& 4. Verify database query performance \\
\hline
\textbf{Expected Result} & All pages load within 3 seconds \\
\hline
\textbf{Actual Result} & Average response time: 1.2 seconds \\
\hline
\textbf{Status} & PASS \\
\hline
\end{tabular}
\caption{Performance Test Case - Response Time}
\end{table}

\section{Test Results Summary}

\begin{table}[h]
\centering
\begin{tabular}{|l|c|c|c|c|}
\hline
\textbf{Test Category} & \textbf{Total Tests} & \textbf{Passed} & \textbf{Failed} & \textbf{Success Rate} \\
\hline
Unit Testing & 15 & 15 & 0 & 100\% \\
\hline
Integration Testing & 8 & 8 & 0 & 100\% \\
\hline
System Testing & 12 & 12 & 0 & 100\% \\
\hline
Performance Testing & 5 & 5 & 0 & 100\% \\
\hline
\textbf{Total} & \textbf{40} & \textbf{40} & \textbf{0} & \textbf{100\%} \\
\hline
\end{tabular}
\caption{Test Results Summary}
\end{table}

\section{Test Environment}

\subsection{Hardware Configuration}
\begin{itemize}
    \item Processor: Intel Core i5-8400 @ 2.8GHz
    \item RAM: 16 GB DDR4
    \item Storage: 512 GB SSD
    \item Network: 100 Mbps Ethernet
\end{itemize}

\subsection{Software Configuration}
\begin{itemize}
    \item Operating System: Ubuntu 20.04 LTS
    \item Java Version: OpenJDK 11
    \item Apache Tomcat: 9.0.65
    \item MySQL: 8.0.30
    \item Browser: Chrome 104.0.5112.79
\end{itemize}

\section{Defect Analysis}

During the testing phase, minor issues were identified and resolved:
\begin{itemize}
    \item Input validation improvements for edge cases
    \item UI responsiveness adjustments for mobile devices
    \item Database query optimization for large datasets
    \item Session timeout handling enhancements
\end{itemize}

All identified defects were successfully resolved before final deployment.